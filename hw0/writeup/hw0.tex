\documentclass{hw}
\usepackage{graphicx}
\usepackage{caption}
\usepackage{subcaption}
\usepackage{float}

\usepackage{xcolor}
\usepackage{enumitem}
\usepackage{listings}
\usepackage{amsmath}% mathtools includes this so this is optional
\usepackage{mathtools}

\def\m{{\textit{m}}}
\def\i{{\textit{i}}}
\def\j{{\textit{j}}}
\def\k{{\textit{k}}}

\newcommand{\hwnum}{0}
\newcommand{\duedate}{September 4, 11:59pm EDT}
\renewcommand{\title}{Python Basics}

\newcommand{\io}{\textbf{Code input and output format.} }
\newcommand{\submission}{\textbf{Submission.}}
\newcommand{\carolina}[1]{\textcolor{purple}{carolina: #1}}
\newcommand{\subham}[1]{\textcolor{orange}{Subham: #1}}

\setboolean{withsolutions}{false}

\newtheorem{claim}{Claim}

%\usepackage{tikz}
%\usetikzlibrary{positioning}
%\tikzstyle{skipnode} = [rectangle, draw, fill=white, minimum width=1cm, minimum height=0.5cm]
\lstset{
  basicstyle=\ttfamily\scriptsize,    % Use a scriptsize typewriter font
  breaklines=true,               % Break long lines
  frame=single,                  % Frame the code block
  numbers=left,                  % Add line numbers on the left
  numberstyle=\tiny\color{gray}, % Make line numbers small and gray
  keywordstyle=\color{blue},     % Color keywords blue
  commentstyle=\color{green},    % Color comments green
  stringstyle=\color{red},       % Color strings red
}
\begin{document}
\newpage

% \section*{Welcome to CS 5112, Algorithms and Data Structures for Applications!}

In this mini-assignment, you are expected to gain a basic understanding of basic pythonic concepts, as well as basic recursion and dynamic programming. This assignment is generated largely with chatGPT, to cover the basics of Python.

Students are expected to solve these problems by browsing the internet to find relevant documentation to enable them to solve the problems. \textbf{We strongly advise against} using chatGPT to solve these basic problems.

\begin{problem}
\textbf{p1\_a:}

In this problem set, we warm-up on:

\begin{subproblem}
If-Elif-Else statements
\begin{enumerate}
    \item categorize\_number
\end{enumerate}
\end{subproblem}

\begin{subproblem}
Enumerating lists, reversing lists.
\begin{enumerate}
    \item find\_index
    \item reverse\_list
\end{enumerate}
\end{subproblem}

\begin{subproblem}
For loops with break and continue statements
\begin{enumerate}
    \item process\_numbers
\end{enumerate}
\end{subproblem}

\begin{subproblem}
Building a basic dictionary
\begin{enumerate}
    \item string\_lengths
\end{enumerate}
\end{subproblem}

\begin{subproblem}
Conditional list comprehensions
\begin{enumerate}
    \item even\_values\_keys
    \item square\_evens
    \item long\_strings
    \item convert\_to\_celsius
\end{enumerate}
\end{subproblem}

\end{problem}

\begin{problem}
\textbf{p1\_b:}

In this problem set, we warm-up on lists. Specifically:

\begin{subproblem}
Appending
\begin{enumerate}
    \item add\_element
\end{enumerate}
\end{subproblem}

\begin{subproblem}
Inserting by index
\begin{enumerate}
    \item insert\_element\_at
\end{enumerate}
\end{subproblem}

\begin{subproblem}
Removing by index
\begin{enumerate}
    \item remove\_element
\end{enumerate}
\end{subproblem}

\begin{subproblem}
Popping from list
\begin{enumerate}
    \item pop\_element\_at
\end{enumerate}
\end{subproblem}

\end{problem}

\begin{problem}
\textbf{p1\_c:}

In this problem set, we warm-up on dictionaries. Specifically:

\begin{subproblem}
Keys, values, items, get
\begin{enumerate}
    \item get\_keys
    \item get\_values
    \item get\_items
    \item get\_value
\end{enumerate}
\end{subproblem}

\end{problem}

\begin{problem}
\textbf{p1\_d:}

In this problem set, we warm-up on:

\begin{subproblem}
Basic list comprehension
\begin{enumerate}
    \item square\_all
    \item filter\_even
\end{enumerate}
\end{subproblem}

\begin{subproblem}
reduce and lambda functions
\begin{enumerate}
    \item product\_of\_all
\end{enumerate}
\end{subproblem}

\end{problem}

\begin{problem}
\textbf{p1\_e:}

In this problem set, we warm-up on:

\begin{subproblem}
Applying reduce-lambda for calculating factorials. Provide a basic write-up describing your solution for this problem.
\begin{enumerate}
    \item factorial\_reduce
\end{enumerate}
\end{subproblem}

\begin{subproblem}
Applying recursion for calculating factorial. Provide a basic write-up describing your solution for this problem.
\begin{enumerate}
    \item factorial\_rec
\end{enumerate}
\end{subproblem}

\end{problem}


\begin{challenge}
\textbf{p1\_advanced:}

In this problem set, we give four advanced problems which warm-up or provide a gentle introduction to:

\begin{subproblem}
Appending, popping, removing and inserting to lists.
\begin{enumerate}
    \item manage\_queue
\end{enumerate}
\end{subproblem}

\begin{subproblem}
Applying dynamic programming for calculating factorial
\begin{enumerate}
    \item factorial\_dp
\end{enumerate}
\end{subproblem}

\begin{subproblem}
Simple linear search problem
\begin{enumerate}
    \item linear\_search
\end{enumerate}
\end{subproblem}

\begin{subproblem}
Binary search problem. Provide a basic write-up describing your solution for this problem, as well as provide a graph comparing the performance of linear\_search and binary\_search for list of the following sizes: [10, 100, 1000, 10000, 100000, 1000000].
\begin{enumerate}
    \item binary\_search
\end{enumerate}
\end{subproblem}

\end{challenge}

\end{document}
